\documentclass{article}
\usepackage[utf8]{inputenc}
\usepackage{geometry}
\usepackage{hyperref}
\usepackage{natbib}
\usepackage{graphicx}
\usepackage{float}


\begin{document}

% Title page: the course code and course name, student name(s) and ID(s), date, the topic of your project.
\begin{titlepage}
 \begin{center}
  \vspace*{1cm}
  \begin{huge}
   \textbf{Real or Not? NLP with Disaster Tweets}
  \end{huge}


  \vspace{0.5cm}
  ELEC-E5550 - Statistical Natural Language Processing

  \vspace{1.5cm}
  Group 1: Anand Umashankar 765837, Swati Choudhary 795920

  \vspace{0.5cm}
  Email : firstname.lastname@aalto.fi

  \vspace{0.5cm}
  21 January 2019
 \end{center}
\end{titlepage}



\section{Introduction}
The aim of the project is to use natural language processing techniques to detect if the tweets are about a real disaster or not. The dataset is obtained from  \href{https://www.kaggle.com/c/nlp-getting-started/overview/description}{Kaggle}. The dataset has the following attributes : tweet text, keywords and location. The data is then subjected to  pre-processing techniques like word2vec, lemmatization, removal of stop words, etc using Python \href{https://www.nltk.org/}{NLTK library}. We plan to use two approaches to predict the class.

\begin{enumerate}
 \item Once the data is preprocessed, we use supervised machine learning classifier like Support Vector Machine(SVM)(This is an initial plan, we might end up using a different classifier).
 \item The second method is to use a Recurrent Neural Network(RNN).
\end{enumerate}
The results of the two techniques are then evaluated to see which technique gives more accurate prediction for the problem in consideration.

\section{Literature Survey}
\section{Dataset}
\section{Preprocessing}

\section{Experiments}
  \subsection{Embeddings}
    \subsubsection{BERT}
    \subsubsection{Glove}
  \subsection{Classifiers}
    \subsubsection{NN}
    \subsubsection{RNN}

\section{Reults}

\section{Conclusions \& Future Work}
Done : Bert embeddings to NN

To Do:
Bert embeddings to RNN
Add twitter glove embeddings
Use the glove embeddings to NN
Use the glove embeddings to RNN
% References: The bibliography that you utilized, and that you refer to in the text
% \bibliographystyle{apalike}
% \bibliography{references}

% \appendix
% \section{Appendix}
% Appendix: Your own program code, possible excerpts of data etc.
% append samples of data set to the Appendix
% I linked the git repo and linked the audio set so I think these are unnecessary


\end{document}
